\section{\D{} Error Processing\label{errorprocess}}
\subsection{The \D{} Internal Error Facility}

\D{} contains a number of in-built error checks scattered
throughout the package which detect a wide range of possible errors.
In all cases, when an error is detected the subroutine {\sc error}
is called, resulting in an appropriate message and termination of the
program execution (either immediately, or after additional
processing.).

Users intending to insert new error checks should ensure that all
error checks are performed {\em concurrently} on {\em all} nodes, and
that in circumstances where a different result may obtain on different
nodes, a call to the global status routine {\sc gstate} is made to set
the appropriate global error flag on all nodes. Only after this is
done, a call to subroutine {\sc error} may be made. An example of such
a procedure might be:

{\tt
\begin{tabbing}
XXXXXX\=\kill
\> logical safe\\
\> safe=({\em test\_condition})\\
\> call gstate(safe)\\
\> if(.not.safe) call error(node\_id,message\_number)\\
\end{tabbing}
}

In this example it is assumed that the logical operation {\em
test\_condition} will result in the answer {\em .true.} if it is safe
for the program to proceed, and {\em .false.} otherwise. The call to
{\sc error} requires the user to state the identity of the calling
node ({\tt node\_id}), so that only the nominated node in {\sc error}
(i.e. node 0) will print the error message. The variable {\tt
message\_number} is an integer used to identify the appropriate
message to be printed. 

In all cases, if {\sc error} is called with a {\em non-negative}
message number, the program run terminates. If the message
number is {\em negative}, execution continues, but even in this case
\D{} will terminate the job at a more appropriate place. This
feature is used in processing the CONTROL and FIELD file directives.
A possible modification users may consider is to dump additional data
before the call to {\sc error} is made.

A full list of the \D{} error messages \index{error messages} and the
appropriate user action can be found in Appendix \ref{A2} of this document.

