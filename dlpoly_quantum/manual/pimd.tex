\section{Overview}
\index{pimd}

Path Integral Molecular Dynamics (PIMD) is an adaptation of Feynmann's
path integral method \cite{feynmann-65} for calculating the properties
of quantum systems. PIMD is derived from an earlier method known as
Path Integral Monte Carlo (PIMC) and is applicable to the same range
of molecular systems. However, despite being a molecular dynamics
technique PIMD is {\em not} an implementation of quantum dynamics and,
unlike classical molecular dynamics, provides {\em no} dynamical
information about the system under study. Properties such as time
dependent correlation functions and transport coefficients are
therefore beyond its scope.  It is purely concerned with obtaining
quantum corrected thermodynamics and structural properties. For this
reason, PIMD is best thought of as PIMC by other means.

PIMD was implemented in \D{} by Bill Smith. Users should note that the
implementation cannot deal with rigid bonds, rigid molecules (or rigid
molecular parts) or polarisable ions. It can however handle free atoms,
rigid ions and fully flexible molecules.

\section{Theory of PIMD}
\index{pimd!theory of}
\label{pimdtheory}

The theory of path integrals is described in a very accessible manner
in the book `Quantum Mechanics and Path Integrals' by Feynmann and
Hibbs \cite{feynmann-65} and a modern take on the subject can be found
in Mark Tuckermann's book `Statistical Mechanics: Theory and Molecular
Simulation' \cite{tuckerman-10}. Both are highly recommended. Aspects
of the theory implemented in \DD{} are described in chapter 12 of the
e-book `Elements of Molecular Dynamics' by Smith \cite{smith-14}. In
this section we outline the essential details only.

The central idea of both PIMD and PIMC is that the partition function
of a system of {\em quantum} particles is isomorphic with that of a
system of {\em classical} ring polymers. In this isomorphism each ring
polymer corresponds to one quantum particle\footnote{As far as \DD{}
  is concerned, a `quantum particle' is a simple atom or
  ion.}. The so-called `beads' of the polymer represent alternative
locations for the quantum particle according to the probabilistic
interpretation of the wave function and collectively describe the
`Heisenberg uncertainty' in the quantum particle's location. The ring
polymer is linked together by harmonic bonds acting between pairs of
beads, which maintain the integrity of the ring and constrain the
beads to locations that are consistent with quantum
probability. Otherwise the beads are essentially free particles
independently exploring phase space.

We can write the partition function ($Z_{N}$) for the N-particle
quantum system as follows (in which the resemblance to a system
composed of classical ring polymers should be evident):
\begin{equation}
  Z_{N}= \left ( \frac{mL}{2 \pi \hbar^{2} \beta} \right )^{3LN/2} \int \ldots \int \exp \left ( -\beta \sum_{j=1}^{N} 
\sum_{k=1}^{L} \left [ \frac{\kappa}{2}(|\vek{ r}_{j,k}-\vek{ r}_{j,k-1}| )^{2}+ 
 \frac{1}{L} \Phi\left(\vek{ r}_{j,k}\right) \right ] \right ) \Pi_{j}^{N} \Pi_{k}^{L} d \vek{ r}_{j,k}, \label{qpartition}
\end{equation}
with
\begin{equation}
  \kappa = \frac{mL}{(\beta \hbar)^{2}}. \label{qspring}
\end{equation}
In this equation $N$ is the number of quantum particles and $L$ is the
number of beads chosen to represent each particle. Each pair of
indices $j$ and $k$ identifies the $j'th$ particle and its $k'th$
bead. Symbols $\Pi_{j}^{N} \Pi_{k}^{L} d \vek{r}_{j,k}$ represent a
product of the differential coordinates of all beads
(i.e. $dx_{j,k}dy_{j,k}dz_{j,k} \ldots etc.$).  The parameter $\kappa$
is the force constant of the harmonic bonds and (\ref{qspring}) shows
that it is defined by the quantum particle's mass, the system
temperature (through Boltzmann's constant $\beta=1/k_{B}T$), Planck's
constant $\hbar$ and the number of beads in the ring $(L)$. The number
$L$ is an adjustable parameter and its value reflects how
`quantum-like' the particle behaves. A low number of beads $(L\sim
10)$ implies the particle behaves semi-classically, while a large
number $L\sim 300$ implies significant quantum behaviour. This gives
some scope for `tuning' the quantum nature of the simulation to suit
the application. Formally, equation (\ref{qpartition}) is exact only
in the limit: $L \rightarrow \infty$, otherwise PIMD should be
regarded as semi-quantitative.

The potential function $\Phi\left (\vek{r}_{j,k}\right)$ appearing in
equation (\ref{qpartition}) describes how an individual bead, with
indices $j$ and $k$, interacts with its environment, and it is
significant from a quantum perspective that it appears in
(\ref{qpartition}) with the associated factor $1/L$.  The most
important quantum effect in atomistic simulations is arguably quantum
tunnelling - the ability of an atom to penetrate regions that are
classically forbidden. This effect can be seen in the radial
distribution function (for example), which describes the atomistic
structure and therfore impacts on the thermodynamics of the system. In
this context the potential factor of $1/L$ indicates that the bead
interaction with the environment is reduced in comparison with the
original classical particle. This allows a greater ability of the beads
to penetrate into classically improbable regions, which is consistent
with the idea of quantum tunnelling.

The energy of the harmonic bonds (determined by the parameter $\kappa$
in (\ref{qpartition})) is a component of the kinetic energy of the
quantum particle related to its zero point energy.  However, in the
classical simulation of the isomorphic ring polymers, this term is
manifestly a contributor to the configurational energy of the
rings. Its meaning is therefore different in the two systems and this
fact should be noted. (Incidentally, note that when the subscript
$k-1$ equals zero in (\ref{qpartition}) this is interpreted as the
index $L$, which is required by the closure of the ring polymer.)

When the environment of the quantum particle consists of other quantum
particles, the form of the potential function $\Phi\left
(\vek{r}_{j,k}\right)$ may (for example) be written as a sum of
bead-bead pair potentials:
\begin{equation}
  \Phi\left (\vek{r}_{j,k}\right )= \frac{1}{2} \sum_{j' \ne j}^{N} \phi_{pair}
  ( |\vek{r}_{j,k}- \vek{r}_{j',k} |), \label{qpair}
\end{equation}
where $\phi_{pair}$ is a pair potential of the van der Waals or
Coulombic kind.  What is noticeable about equation (\ref{qpair}) is
that there is no sum over the bead index $k$ on the right. This
reflects the fact that, due to a quirk of the path integral origins of
(\ref{qpair}) \cite{smith-14}, a given bead on the polymer ring ($k$)
can only interact with a single corresponding bead ($k'$) on
neighbouring polymer rings i.e. $k=k'$ and interactions where $k'\ne
k$ are forbidden\footnote{Technically speaking, beads $k$ and $k'$
  exist at different instants of imaginary time and are therefore
  invisible to each other.}. This requirement means the calculation of
the potential energy and force is considerably less costly than would
be expected in a simulation of true classical ring polymers. This
amounts to a substantial `decoupling' of the bead-bead interactions
(in classical terms) and means that, to a large extent, the whole
system can be simulated as a set of $L$ quasi-independent systems,
which are coupled via the harmonic springs. This decoupling is
exploited in the way \DD{} performs the PIMD simulations.

In principle, the simulation of a system of ring polymers with harmonic
bonds, bead masses and bead-bead interactions as defined above, will
effectively be a simulation of the corresponding quantum system, correct to
an accuracy determined by the chosen number of beads. From such a
simulation the structure and thermodynamics of the quantum system can
be obtained in a manner that is similar to any classical simulation.
It is evident that the simulation could equally be performed using
either molecular dynamics or Monte Carlo, though in the case of \DD{},
it is inevitably the former choice that is taken.

A potential problem with PIMD simulations is that harmonic bonds are
notoriously non-ergodic, which means that energy exchange between
degrees of freedom is inefficient and the system is slow to
equilibrate. Indeed, even very long simulations can fail to achieve
the equipartition necessary for a successful calculation of
thermodynamic properties {\em unless strong preventative measures are
  taken}.  As described in \cite{tuckerman-10} these measures include
the use of staging variables in place of the usual atomic coordinates
and an extensive regime of thermostatting using either Nos\'{e}-Hoover
chains \cite{martyna-92a,tuckerman-10} or the so-called `gentle'
thermostats of Leimkuhler {\em et al} \cite{leimkuhler-09a}, both of
which which are options in the \DD{} implementation.  Note that,
according to (\ref{qspring}), a large particle mass $m$, or a large
number of beads $L$, or a high temperature (i.e. small $\beta$)
implies a large force constant $\kappa$.  Large values of $\kappa$
worsen the ergodicity problems and simultaneously demand an unusually
short time step for stable simulations, which leads to slower sampling of
phase-space. It is therefore usual to decrease the number of beads
when the particle mass is large or the simulation temperature is high,
since the quantum particles behave more classically in these cases.

In PIMD molecular dynamics is used entirely as a means to explore the
phase space of the system. (In PIMC the same job is done by Metropolis
sampling.) The dynamics is based on a Hamiltonian of the form
\begin{equation}
H(\{\vek{p}_{j,k}\},\{\vek{r}_{j,k}\})=\sum_{j=1}^{N} \sum_{k=1}^{L}
\left [ \frac{p^{2}_{j,k}}{2m'}+ \frac{\kappa}{2}(|\vek{r}_{j,k}-\vek{
    r}_{j,k-1}| )^{2}+ \frac{1}{L} \Phi \left (\vek{ r}_{j,k} \right ) \right ], \label{qhamil}
\end{equation} 
in which we have introduced the bead momentum $\vek{p}_{j,k}$ and a
so-called bead dynamical mass $m'$, which need not be the same as the
formal mass $m$ of the quantum particle if the intention is merely to
use this Hamiltonian to drive the dynamics of the classical system.
Suitable equations of motion are easily obtained from this. However,
as was mentioned above, it is essential to augment the dynamics with
methods to ensure ergodic behaviour. This is described in greater
detail below, following a description of properties calculable by
PIMD.

An important thermodynamic property of a quantum system is the
internal energy $E$, which is given thermodynamically as the
derivative of the partition function:
\begin{equation}
  \left \langle E \right \rangle =-\frac{\partial}{\partial \beta} ln (Z_{N}). 
\end{equation}
Applying this rule to the partition function (\ref{qpartition}) leads
to the standard energy estimator for PIMD:
\begin{equation}
  \left \langle E \right \rangle =\frac{3LN}{2 \beta}- \left \langle
  \sum_{j=1}^{N} \sum_{k=1}^{L} \left [ \frac{\kappa}{2} (| \vek{ r}_{j,k}-\vek{
      r}_{j,k-1}| )^{2} -\frac{1}{L} \Phi\left (\vek{ r}_{j,k} \right ) \right ]  \right \rangle .
  \label{Est1}
\end{equation}
However, it was shown by Herman {\em et al.} \cite{herman-82a} that
fluctuations in this estimator grow linearly with the number of beads
$L$, raising concerns about the accuracy of the mean value. An
alternative estimator based on the virial was offered that overcame this
deficiency. This is written as
\begin{equation}
  \left \langle E \right \rangle =\frac{3N}{2 \beta}+ \left \langle \frac{1}{L}
  \sum_{j=1}^{N} \sum_{k=1}^{L} \left [ \frac{1}{2} (\vek{ r}_{j,k}-\vek{
      R}_{j} )\cdot \frac{\partial}{\partial \vek{r}_{j,k}}\Phi \left (\vek{ r}_{j,k} \right )+
    \Phi \left (\vek{ r}_{j,k} \right )\right ]  \right \rangle , \label{Est2}
\end{equation}
where
\begin{equation}
  \vek{R}_{j}=\frac{1}{L} \sum_{k=1}^{L} \vek{r}_{j,k}, \label{qcom}
\end{equation}
is the centre of mass of the $j'th$ ring. The virial estimator
(\ref{Est2}) ought to be more accurate than the standard energy
estimator (\ref{Est1}), but in practice this is not always the
case. Both versions are available in \DD{} and a comparison between the
two is useful as an indicator of statistical convergence.

A formula for the system pressure in PIMD simulations can be obtained
by differentiating the Hamiltonian (\ref{qhamil}) with respect to volume:
\begin{equation}
  P=-\left \langle \left ( \frac{\partial}{\partial V}H(\{\vek{p}_{j,k}\},\{\vek{r}_{j,k}\})
     \right ) \right \rangle .
\end{equation}
Assuming exclusively pair potentials for the bead-bead interactions leads to
\begin{equation}
  PV=\frac{LN}{\beta}-\frac{1}{3} \left \langle \sum_{j=1}^{N} \sum_{k=1}^{L}
  \left [ \kappa (| \vek{r}_{j,k}-\vek{r}_{j,k-1} |)^{2}+\frac{1}{2L} \sum_{j' \ne j}^{N}
    \left ( \frac{\partial}{\partial \vek{r}_{jj',k}}\phi_{pair}(\vek{ r}_{jj',k}) 
    \right )\cdot \vek{ r}_{jj',k}) \right ] \right \rangle .
\end{equation}
where
\begin{equation}
  \vek{r}_{jj',k}=\vek{r}_{j',k}-\vek{r}_{j,k} .
\end{equation}

With regard to structural properties, the radial distribution function
$g(r)$ can be obtained in a manner closely resembling the classical
case (allowing for the multiplicity of beads per atom):
\begin{equation}
  g(r)=\frac{1}{LN \rho} \left \langle \sum_{k=1}^{L} \sum_{j=2}^{N} \sum_{j'=1}^{j}
  \delta(|\vek{r}-\vek{r}_{j',k}+\vek{r}_{j,k}|) \right \rangle .
\end{equation}
From this formula it can be seen that $g(r)$ is calculated entirely
from correlations between corresponding beads $k$ on different ring
polymers.

Another useful structural property is the mean-square radius of
gyration of the polymer rings $\langle R^{2} \rangle$ . This provides a
measure of the {\em delocalisation} of the quantum particle implied by
Heisenberg uncertainty. It is defined as
\begin{equation}
  \left \langle R^{2} \right \rangle = \left \langle \frac{1}{LN}
  \sum_{j=1}^{N} \sum_{k=1}^{L} | \vek{r}_{j,k}-\vek{R}_{j} |^{2} \right \rangle
\end{equation}
where $\vek{R}_{j}$ is the centre of mass of the $j'th$ polymer ring
and is given by (\ref{qcom}).

The mean-square radius is related to the mean-square bondlength $
\langle B^{2} \rangle $ defined as
\begin{equation}
  \langle B^{2} \rangle =\left \langle |\vek{r}_{j+1,k}-\vek{r}_{j,k}|^{2} \right \rangle
\end{equation}
(where the index $j+1 \equiv 1$ when $j=L$.) The mean-square
radius and bondlength are related through the expression
\begin{equation}
  \left \langle R^{2} \right \rangle =\frac{L}{12} \langle B^{2} \rangle,
\end{equation}
which holds for ideal ring polymers in equilibrium. This relation
affords a rough check that the system is properly equilibrated, along
with other indicators of equilibrium, which are described below.

\section{Path Integral Dynamics}
\index{pimd!dynamics}
\label{pimddynamics}
In the main, the implementation of PIMD in \D{} follows the
prescription given by Tuckerman {\em et al} \cite{tuckerman-93a}. In
the first part, the Cartesian coordinates defining the positions of
the polymer beads are cast into collective coordinates better able to
address the range of time scales in the motion of the rings. This
process is known as `staging'. In the second part the dynamics of the
staged coordinates are subjected to an efficient thermostatting regime
to ensure the system dynamics are properly ergodic.

The staging coordinates $\vek{u}_{j,k}$  are defined in the following manner:
\begin{eqnarray}
  \vek{u}_{j,1}&=&\vek{r}_{j,1}  \nonumber \\
  \vek{u}_{j,k}&=&\vek{r}_{j,k}-\frac{(k-1)\vek{r}_{j,k+1}+\vek{r}_{j,1}}{k}\phantom{xxxxx} (k=2,\ldots,L),
\end{eqnarray}
for which the inverse is obtained using the reverse formula:
\begin{eqnarray}
  \vek{r}_{j,1}&=&\vek{u}_{j,1}  \nonumber \\
  \vek{r}_{j,k}&=&\vek{u}_{j,k}-\frac{(k-1)\vek{r}_{j,k+1}+\vek{r}_{j,1}}{k}\phantom{xxxxx} (k=L,\ldots,2).
\end{eqnarray}

In addition, a set of parameters $\mu_{k}$ is introduced, which is defined as follows:
\begin{eqnarray}
  \mu_{1}&=&0 \nonumber \\
  \mu_{k}&=&\frac{k}{k-1}\phantom{xxxxxxxxxxxxx} (k=2,\ldots,L).
\end{eqnarray}

In terms of these variables the driving Hamiltonian (\ref{qhamil}) can be re-written as
\begin{equation}
  H(\{\vek{\pi}_{j,k}\},\{\vek{u}_{j,k}\})=\sum_{j=1}^{N} \sum_{k=1}^{L}
  \left [ \frac{\pi^{2}_{j,k}}{2m_{k}'}+\mu_{k}\frac{\kappa}{2}(\vek{u}_{j,k}
    \cdot\vek{u}_{j,k} )+ \frac{1}{L} \Phi \left (\vek{r}_{j,k}(\{\vek{u}_{j',k}\} \right )) \right ],
  \label{qhamil2}
\end{equation}
where $\vek{\pi}_{j,k}$ represents the momentum associated with the
staged coordinate $\vek{u}_{j,k}$.  In this equation a new mass
parameter $m_{k}'$ has been introduced as the dynamical mass. The
recommended values given by Tuckerman {\em et al.} \cite{tuckerman-93a}
are
\begin{eqnarray}
  m_{1}'&=&m \nonumber \\
  m_{k}'&=& \mu_{k} m \phantom{xxxx}(k>1).
\end{eqnarray}
Using the Hamiltonian (\ref{qhamil2}) we may proceed to obtain the
equations of motion for the system of ring polymers. These are
presented below with the necessary thermostats included. The
thermostats, it should be noted, are applied to {\em every} degree of
freedom in the system individually - not collectively as is more
common in conventional molecular dynamics.

The equations of motion including Nos\'{e}-Hoover chains of length $M$
per bead degree of freedom \cite{tuckerman-93a} are
\begin{eqnarray}
  \vek{\dot{u}}_{j,k}&=&\frac{\vek{\pi}_{j,k}}{m_{k}'} \nonumber \\
  {\dot \pi}_{j,k}^{\alpha}&=&-\kappa \mu_{k}u_{j,k}^{\alpha}-\frac{1}{L} \frac{\partial}
  {\partial u_{j,k}^{\alpha}}\Phi\left (\vek{r}_{jk}(\{\vek{u}_{j',k}^{\alpha}\})\right )-
  \zeta_{j,k,1}^{\alpha}{\pi}_{j,k}^{\alpha} \nonumber \\
  {\dot \zeta}_{j,k,1}^{\alpha}&=&\frac{1}{Q_{1}} \left (\frac{(\pi_{j,k}^{\alpha})^{2}}{m_{k}'}-k_{B}T_{o} \right )
  -\zeta_{j,k,1}^{\alpha}\zeta_{j,k,2}^{\alpha} \label{pimd-nhc}\\
  {\dot \zeta}_{j,k,n}^{\alpha}&=&\frac{1}{Q_{k}} \left (Q_{k}(\zeta_{j,k,n-1}^{\alpha})^{2}-k_{B}T_{o} \right )
  -\zeta_{j,k,n}^{\alpha}\zeta_{j,k,n+1}^{\alpha}  \nonumber\\
  {\dot \zeta}_{j,k,M}^{\alpha}&=&\frac{1}{Q_{k}} \left (Q_{k}(\zeta_{j,k,M-1}^{\alpha})^{2}-k_{B}T_{o} \right ) \nonumber
\end{eqnarray}
where $\alpha=x,y,z$ as appropriate and $T_{o}$ is the target
temperature.  The variables $\zeta_{j,k,n}$ {\em etc.} are the
thermostat variables. The values of the thermostat `mass' parameters
$Q_{k}$ recommended by Tuckerman {\em et al} \cite{tuckerman-93a} are
\begin{equation}
  Q_{1}=k_{B}T_{o} \tau^{2}, \phantom{xxxxx} Q_{k}=\hbar^{2}/Lk_{B}T_{o},
\end{equation}
where $\tau$ is a user defined time scale parameter appropriate for
the timescale of the ring centre of mass motion..

The forces required to update the momentum components
${\pi}_{j,k}^{\alpha}$ above are obtained from
\begin{eqnarray}
  \frac{\partial}{\partial u_{j,1}^{\alpha}}\Phi \left (\vek{r}_{jk}(\{\vek{u}_{j',k}^{\alpha}\}) \right )&=&
  \frac{1}{L}\sum_{k=1}^{L}\frac{\partial}{\partial r_{j,k}^{\alpha}}\Phi\left (\vek{r}_{jk} \right ), \nonumber \\
  \frac{\partial}{\partial u_{j,k}^{\alpha}}\Phi\left (\vek{r}_{jk}(\{\vek{u}_{j',k}^{\alpha}\}) \right )&=&
  \frac{1}{L}\left ( \frac{(k-2)}{(k-1)}\frac{\partial}{\partial u_{j,k}^{\alpha}}+
  \frac{\partial}{\partial r_{j,k}^{\alpha}} \right ) \Phi\left (\vek{r}_{jk}(\{\vek{u}_{j',k}^{\alpha}\}) \right ).
\end{eqnarray}
In \DD{} these equations are integrated using the velocity Verlet
algorithm and it is useful to note that the following quantity is
formally conserved during the integration.
\begin{equation}
  E_{cons}=\sum_{j=1}^{N}\sum_{k=1}^{L} \left [ \frac{\pi_{j,k}^{2}}{2 m_{k}'}+
    \mu_{k} \frac{\kappa}{2} (\vek{u}_{j,k} \cdot \vek{u}_{j,k})+
    \frac{1}{L}\Phi\left (\vek{r}_{jk}(\{\vek{u}_{j',k}^{\alpha}\}) \right )+ \newline
    \sum_{n=1}^{M} \sum_{\alpha} \left \{ \frac{1}{2}Q_{k}({\dot \zeta}_{j,k,n}^{\alpha})^{2}
    +\frac{1}{\beta}\eta_{j,k,n}^{\alpha} \right \} \right ]. \label{qcons}
\end{equation}
The variables $\eta_{j,k,n}$ are obtained by integrating the variables
$\zeta_{j,k,n}$ using the equation
\begin{equation}
  {\dot \eta}_{j,k,n}^{\alpha}=\zeta_{j,k,n}^{\alpha}.
\end{equation}
The last sum on the right of (\ref{qcons}) clearly represents the
energy of the thermostats, while the preceeding terms refer to the quantum system.

The corresponding equations of motion with the gentle thermostats
of Leimkuhler {\em et al} \cite{leimkuhler-09a} (which is again
applied to every degree of freedom) are
\begin{eqnarray}
  \vek{\dot{u}}_{j,k}&=&\frac{\vek{\pi}_{j,k}}{m_{k}'} \nonumber \\
  {\dot \pi}_{j,k}^{\alpha}&=&-\kappa \mu_{k}u_{j,k}^{\alpha}-\frac{1}{L} \frac{\partial}
  {\partial u_{j,k}^{\alpha}}\Phi\left (\vek{r}_{jk}(\{\vek{u}_{j',k}^{\alpha}\})\right )-
  \zeta_{j,k}^{\alpha}{\pi}_{j,k}^{\alpha}   \label{pimd-gt} \\
  d{\zeta}_{j,k}^{\alpha}&=&\frac{1}{Q} \left (\frac{(\pi_{j,k}^{\alpha})^{2}}{m_{k}'}-k_{B}T_{o} \right )dt
  -\frac{Q\chi^{2}}{2 k_{B}T_{o}}\zeta_{j,k}^{\alpha}dt +\chi dW. \nonumber 
\end{eqnarray}
The first two equations in this series determine the thermostatted
motion of the beads ({\em c.f.} (\ref{pimd-nhc})), while the third
determines the motion of the thermostats $\zeta_{j,k}$. On the far
right the third equation includes an impulse term $\chi dW$ for each
thermostat, where $\chi$ defines the impulse magnitude and $dW$ is a
random number selected from Gaussian distribution with zero mean and
unit standard deviation. Preceding this on the right is the
dissipation term, which extracts heat energy from the system to
maintain the average temperature. $\chi$ is a user-selected
parameter chosen to control the rate at which energy is put into the
system. The constant of dissipation $Q \chi^{2}/2 k_{B}T_{o}$ is
fixed by the fluctuation-dissipation theorem. In its original
conception \cite{leimkuhler-09a}, the gentle thermostat was intended
to have a minimal effect on the system dynamics, so $\chi$ was
small, which meant that time dependent properties could be preserved,
but that need not be a concern in PIMD simulations, which are
concerned with structure only.

In both approaches (\ref{pimd-nhc}) and (\ref{pimd-gt}), it is useful
to have some means of monitoring the simulation to ensure that it
properly samples from the canonical ensemble. This requires that the
kinetic energy of every degree of freedom has an average value of
$k_{B}T_{o}/2$ and that the staged momenta of the beads are
collectively represented by a Gaussian distribution. In \DD{} the
quality of the obtained momentum distribution is assessed by comparing
the even moments of the simulated distribution function with the ideal
values for a true Gaussian (see \cite{smith-14} ch. 12).  The expected
values for true Gaussian moments are
\begin{eqnarray}
  \left \langle (\pi_{j,k}^{2}/m_{k}') \right \rangle &=& k_{B}T_{o} \nonumber \\
  \left \langle (\pi_{j,k}^{2}/m_{k}')^{2} \right \rangle &=& 3 (k_{B}T_{o})^{2} \nonumber \\
  \left \langle (\pi_{j,k}^{2}/m_{k}')^{3} \right \rangle &=& 15(k_{B}T_{o})^{3}  \label{moments}\\
  \left \langle (\pi_{j,k}^{2}/m_{k}')^{4} \right \rangle &=& 105(k_{B}T_{o})^{4} \nonumber \\
  \left \langle (\pi_{j,k}^{2}/m_{k}')^{5} \right \rangle &=& 945(k_{B}T_{o})^{5}. \nonumber
\end{eqnarray}
Note that, in a simulation, the averages $\langle \ldots \rangle$ are
calculated over all momenta $\vek{\pi}_{j,k}$ and over many
configurations of the equilibrated simulation.  From (\ref{moments})
it follows that, in the ideal case
\begin{equation}
  1=\frac{\left \langle (\pi_{j,k}^{2}/m_{k}') \right \rangle }{k_{B}T_{o}}=
  \frac{\left \langle (\pi_{j,k}^{2}/m_{k}')^{2} \right \rangle }{3 (k_{B}T_{o})^{2}}=
  \frac{\left \langle (\pi_{j,k}^{2}/m_{k}')^{3} \right \rangle }{15 (k_{B}T_{o})^{3}}=
  \frac{\left \langle (\pi_{j,k}^{2}/m_{k}')^{4} \right \rangle }{105 (k_{B}T_{o})^{4}}=
  \frac{\left \langle (\pi_{j,k}^{2}/m_{k}')^{5} \right \rangle }{945 (k_{B}T_{o})^{5}}.
  \label{nmom}
\end{equation}
Thus the closeness of these ratios to the value unity indicates the
quality of the assumed Gaussian distribution for all momenta.  Note
the same analysis may also be conducted for the thermostat momenta and
this can be useful when exploring ergodicity issues. This is not
available in the default version of \D{}, but a subroutine for doing
this ({\sc thermostat\_moments}) is available in the module {\sc
  integrator\_module.f} and can be activated for the purpose.

\section{Invoking the PIMD Option}

To activate the PIMD option in \D{} appropriate entries must be made
in the CONTROL file. At the head of the CONTROL file, in place of the
{\bf integration} keyword, the keyword {\bf pimd} should be inserted,
followed by additional options. The syntax is one of the following
three options\newline
\newline
\noindent {\bf pimd nvt} {\em nbeads taut} \newline
\newline
\noindent for PIMD with a single Nos\'{e}-Hoover thermostat per degree of freedom, or \newline
\newline
\noindent {\bf pimd gth} {\em nbeads taut xsi} \newline
\newline
\noindent for PIMD with `gentle' thermostats, or \newline
\newline
\noindent {\bf pimd nhc} {\em nbeads nchain taut} \newline
\newline    
\noindent for PIMD with Nos\'{e}-Hoover chains. \newline
\newline
The additional options appearing on the directives are:
\begin{itemize}
\item {\em nbeads} (integer) is the number of beads in each ring polymer.
\item {\em nchain} (integer) is the number of chained thermostats per degree of freedom.
\item {\em taut} (real) is the thermostat relaxation time $(\tau)$ in $ps$.
\item {\em chi} (real) is the magnitude of the impulse `force' $(\chi)$ in $ps^{-1}$
\end{itemize}
\noindent Each of these must be written in the indicated order. It is
permissible to use more informative invocations of these directives
({\em provided the sequence order given above is preserved}), such
as \newline
\newline
\noindent {\bf pimd nhc} nbeads=10 nchain=4 taut=0.1  \newline
\newline
\noindent These simple directives are sufficient to invoke a PIMD simulation.
However, the following points should also be noted.
\begin{itemize}
\item As is mentioned in the theory section above, the number of beads
  per ring polymer is a choice based on the expected degree of quantum
  behaviour of the constituent atoms. Some experience is required to
  make an appropriate choice, but note that the same number of beads
  represents {\em all} the atoms in the system, regardless of the
  different atoms that may be present.  The choice must therefore be
  made on the basis of the smallest atomic mass in the system if the
  quantum behaviour is to be properly handled.

\item The PIMD option implies velocity Verlet integration. The user
  should {\bf not} therefore specify the {\bf integration} keyword at
  the head of the CONTROL file. If one is specified, \DD{} will
  {\bf terminate} due to conflicting instructions.

\item PIMD also implies the NVT ensemble, so the user should {\bf not} use
  the {\bf ensemble} keyword in the CONTROL file. {\bf Termination} will
  again result if this is specified.
\end{itemize}

\section{PIMD Files}

In addition to the usual files \DD{} requires or creates, the PIMD
option also makes use of the following files.

\subsection{THENEW and THEOLD}

The file THENEW is created when a PIMD simulation completes. It
contains the current values of all the thermostat variables at the
time of completion. The data is required for a subsequent continuation
of the simulation using the {\bf restart} directive in the CONTROL
file. For this purpose the THENEW file must be renamed as THEOLD at
the restart. These files are {\em unformatted} and not human
readable. The contents, should the user need to know, are as follows.\\

\noindent {\bf Record 1} lists three variables:

\begin{itemize}
  \item Total number of beads in the system i.e. $natms \times nbeads$;
  \item Length of thermostat chain per bead i.e. the variable $nchain$;
  \item The system temperature i.e. that variable $temp$.
\end{itemize}

The subsequent records contain the thermostat chains. There are
$nchain \times natms \times nbeads$ records, each of which contains the following.

\noindent {\bf Records $2$ to $nchain \times natms \times nbeads+1$} list
six variables per record:

\begin{itemize}
  \item $\eta_{x}$ - x component of integrated thermostat of one bead 
  \item $\eta_{y}$ - y component of integrated thermostat of one bead 
  \item $\eta_{z}$ - z component of integrated thermostat of one bead 
  \item $\zeta_{x}$ - x component of thermostat of one bead
  \item $\zeta_{y}$ - y component of thermostat of one bead
  \item $\zeta_{z}$ - y component of thermostat of one bead
\end{itemize}

The order in which the records are written is determined by three
nested loops, the first (outer) loop runs over the $nchain$ thermostat
chains, the second loop runs over the $natms$ atoms and the third
(inner) loop runs over the $nbeads$ beads for each atom.

\subsection{RNDNEW and RNDOLD}

The file RNDNEW is created when a PIMD simulation employing the gentle
thermostats completes. It contains the current values of all the
variables defining the state of the \DD{} parallel random number
generator {\sc puni()} at the time of completion. The information is
required to restart a simulation using the {\bf restart} directive in
the CONTROL file. The RNDNEW file must be renamed as RNDOLD at the
restart. These files are also {\em unformatted} and not human
readable.

For a simulation performed on a parallel computer with $mxnode$
processors, the files contain $mxnode$ records, each with 102
variables defining the state of {\sc puni()} on each processor.

\section{Things to be aware of when running a PIMD simulation}

\begin{itemize}
\item Usually, when starting a \DD{} simulation, the user supplies a
  CONFIG file for the system of interest containing the initial
  coordinates of every atom in the system. For the first run of a PIMD
  simulation however, it is not necessary to supply the coordinates
  for every bead of the ring polymers. Instead the user supplies the
  coordinates for each corresponding classical particle, which \DD{}
  expands using small random displacements into {\em nbeads} coordinates
  to complete the ring. This represents a first approximation to the
  ring polymers, which evolve dynamically thereafter.

\item After the first run of the simulation, the resulting REVCON file
  will contain coordinates for {\em all} the beads in the system. It
  follows that the CONFIG file used to {\bf restart} a simulation
  necessarily contains coordinates for every bead and \DD{} expects
  these data to be present in the file or else the restart will
  fail. This requirement also applies to the {\bf restart scale} and
  {\bf restart noscale} options.

\item The format of the expanded CONFIG file is essentially the same
  as a normal CONFIG file. If there are formally $N$ atoms (i.e. $N$
  quantum particles) in the system and each is represented by $L$
  beads, the expanded file contains $N\times L$ bead coordinates. The
  first $N$ coordinates refer to the first bead of each ring, the
  second set of $N$ coordinates refer to the second bead of each ring,
  and so on. This is important to know if the CONFIG file is to be
  used properly for non-\DD{} applications.

\item In addition to producing the standard output files: OUTPUT,
  REVCON, STATIS, REVIVE, RDFDAT {\em etc}, the PIMD option also
  produces a {\em thermostats} file called THENEW \index{THENEW file}
  (the format of which is described above). This file contains the
  current values of the thermostat parameters at the time the run
  finished. This file should be renamed THEOLD \index{THEOLD file} for
  a {\bf restart} of the PIMD simulation. The file THEOLD is not
  required for the {\bf restart scale} or {\bf restart noscale}
  options.

 \item The {\bf pimd gth} option also produces a thermostat file
   THENEW \index{THENEW file} on termination (and requires the
   corresponding THEOLD file for a restart). In addition however, it
   also produces a file RNDNEW, \index{RNDNEW} which stored the
   current state of the \DD parallel random number generator {\sc
     puni}. This must be renamed RNDOLD \index{RNDOLD} for a
   successful {\bf restart} of the simulation. It is important to
   note that in the current implementation it is assumed the restart
   will require the same number of parallel processors as the previous
   run. The file RNDOLD is not required for the {\bf restart scale} or
   {\bf restart noscale} options.

\item Equilibration and ergodicity are key issues in PIMD
  simulations. A good deal of effort is required sometimes to ensure
  that these requirements are properly fulfilled. Fortunately there
  are some indicators to assist with these matters. Firstly, the
  OUTPUT file lists the second, fourth, sixth, eighth and tenth
  moments (equation \ref{nmom}) of the momentum at the end of the
  simulation. If the momentum distribution is properly ergodic, the
  normalised moments should all be unity. In practice however, the
  higher moments tend to deviate increasingly from unity, though the
  agreement should still be close. Significant departure should be
  viewed critically. Also, if equilibration is good the standard
  \ref{Est1} and virial \ref{Est2} estimators should give closely
  similar results. If they don't then it is likely that ergodicity is
  an issue. The remedy is largely a matter of experimenting with the
  control parameter values for the {\bf pimd} directive or setting a
  shorter time step. To be more certain, using a different
  thermostatting method should afford a good comparison.

\item Two test cases are available to try out the \DD{} PIMD
  implementation (see Chapter \ref{data}). These systems are (a)
  neon (Test Case 40) and (b) water (Test Case 41), which both show
  significant quantum effects but are quite different in character.
  The first is an atomic system, while the latter includes the water
  as a polyatomic molecular entity. Both are outlined in the following
  chapter and both are available from the CCPForge repository, where
  the source code is located. It is worth trying different
  thermostatting methods in these systems from the one supplied.
\end{itemize}


