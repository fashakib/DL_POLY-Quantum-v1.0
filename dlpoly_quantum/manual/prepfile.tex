
\section{A Guide to Preparing Input Files}

The CONFIG file and the FIELD file can be quite large and unwieldy
particularly if a polymer or biological molecule is involved in the
simulation. This section outlines the paths to follow when trying to
construct files for such systems. The description of the \D{} force
field in chapter \ref{field} is essential reading. The various utility
routines mentioned in this section are described in greater detail in
chapter \ref{utility}. Many of these have been incorporated into the
\D{} Graphical User Interface\index{Graphical User Interface}
\cite{smith-gui} and may be convienently used from there.

\subsection {Inorganic Materials}

The utility {\sc genlat} can be used to construct the CONFIG file for
relatively simple lattice structures. Input is interactive. The FIELD
file for such systems are normally small and can be constructed by
hand. The utility {\sc genlat.to} constructs the CONFIG file for
truncated-octahedral boundary conditions. Otherwise the input of force
field data for crystalline systems is particularly simple, if no
angular forces are required (notable exceptions to this are zeolites
and silicate glasses - see below). Such systems require only the
specification of the atomic types and the necessary pair forces. The
reader is referred to the description of the \D{} FIELD file
for further details (section \ref{fieldfile}).

\D{} allows the simulation of zeolites and silicate (or
other) glasses. Both these materials require the use of angular forces
to describe the local structure correctly. In both cases the angular
terms are included as {\em three body terms}, the forms of which are
described in chapter \ref{field}. These terms are entered into the FIELD file
with the pair potentials. Note that you cannot use truncated
octahedral or rhombic dodecahedral boundary conditions\index{boundary
conditions} in conjunction with three body\index{potential!three-body}
forces, due to the use of the link-cell algorithm for evaluating the forces.

An alternative way of handling zeolites is to treat the zeolite
framework as a kind of macromolecule (see below). 
Specifying all this is tedious and is best done
computationally: what is required is to determine the nearest image
neighbours of all atoms and assign appropriate bond and valence angle
potentials. (This may require the definition of new bond forces in
subroutine {\sc bndfrc}, but this is easy.) What must be avoided at
all costs is specifying the angle potentials {\em without} specifying
bond potentials\index{potential!bond}. In this case \D{} will
automatically cancel the non-bonded\index{potential!nonbonded} forces
between atoms linked via valence angles\index{potential!valence angle}
and the system will collapse. The advantage of this method is that the
calculation is likely to be faster using
three-body\index{potential!three-body} forces. This method is not
recommended for amorphous systems.

\subsection{Macromolecules}

Simulations of proteins are best tackled using the package DLPROTEIN
\cite{dlprotein}\index{DLPROTEIN} which is an adaptation of DL\_POLY
specific to protein modelling. However you may simulate proteins and
other macromolecules with \D{} if you wish. This is described below.

If you select a {\em protein} structure from a SEQNET file (e.g.  from
the Brookhaven database), use the utility {\sc proseq} to generate the
CONFIG file. This will then function as input for \D{}. Some caution is
required here however, as the protein structure may not be fully
determined and atoms may be missing from the CONFIG file.

There are further (somewhat limited) tools for processing proteins in
the {\em MACROMOL} subdirectory of the \D{} {\em utility} directory.
These utilities will allow you build a biological system and
create the necessary FIELD file from the
GROMOS\index{force field!GROMOS}\index{GROMOS} \cite{gunsteren-87a} or
AMBER\index{force field!AMBER}\index{AMBER} \cite{weiner-86a} force
fields.

\subsection{Adding Solvent to a Structure}

The utility {\sc wateradd} adds water from an equilibrated
configuration of 256 SPC water molecules at 300 K to fill out the MD
cell. The utility {\sc solvadd} fills out the MD box with single-site
solvent molecules from a f.c.c lattice. The FIELD files will then need
to be edited to account for the solvent molecules added to the file.

\subsection{Analysing Results}

\D{} is not designed to calculate every conceivable property you
might wish from a simulation. Apart from some obvious thermodynamic
quantities and radial distribution functions, it does not calculate
anything beyond the atomic trajectories on-line. You must therefore be
prepared to post-process the HISTORY file if you want other
information. There are some utilities in the \D{} package to help
with this, but the list is far from exhaustive. In time, we hope
to have many more. Users are invited to submit code to the \D{} {\em
public} library to help with this.

Users should be aware that many of these utilities are incorporated
into the DL\_POLY Graphical User Interface\index{Graphical User
Interface} \cite{smith-gui}.
