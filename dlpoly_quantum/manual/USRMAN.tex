\documentclass[11pt,a4paper,dvipdfmx]{report}
\setcounter{secnumdepth}{5}


%packages
\usepackage{makeidx}
\usepackage{comment}
\usepackage[dvips]{graphicx}
\usepackage[usenames,dvipsnames]{color}
\usepackage
[dvipdfmx,
%------------- Back referencing Info --------------------
pagebackref, %or backref
%------------ Doc View ----------------------------------
bookmarksopen=true,
bookmarks=true,
pdfpagemode=UseOutlines,
pdfstartpage=1,
pdfstartview=FitH,
%------------- Doc Colours ------------------------------
colorlinks=true,
filecolor=webbrown, %defined below
citecolor=webgreen, %defined below
linkcolor=webblue, %defined below
urlcolor=webblue, %defined below
%------------- Doc Info ---------------------------------
pdftitle={The DL-POLY Classic User Manual},
pdfauthor={W. Smith},
pdfsubject={Guide to the DL-POLY-Classic Package},
pdfkeywords={Molecular Dynamics (MD) Software, Open Source Code},
]{hyperref}
%
%Define some eye-pleasing colors for this document
%
\definecolor{webgreen}{rgb}{0,0.5,0}
\definecolor{webbrown}{rgb}{0.6,0,0}
\definecolor{webblue}{rgb}{0,0,0.75}

% Layout settings
\setlength{\textheight}{236mm} \setlength{\textwidth}{165mm}
\setlength{\topmargin}{-7mm} \setlength{\headsep}{10mm}
\setlength{\footskip}{10mm} \setlength{\oddsidemargin}{-2mm}
\setlength{\evensidemargin}{-2mm}

%pagestyle
%\pagestyle{myheadings}
%\markboth{\copyright STFC}{\copyright STFC}
\usepackage{fancyhdr}
\pagestyle{fancy}
\lhead{\copyright STFC}
\chead{}
\rhead{Preface}
\lfoot{}
\cfoot{\thepage}
\rfoot{}

%paragraph settings
%\parindent 0ex
%\parskip 1ex

%commands
\newcommand{\D}{\mbox{DL\_POLY Classic}}
\newcommand{\DD}{\mbox{DL\_POLY}}
\newcommand{\DLF}{\mbox{DL\_FIELD}}
\newcommand{\SOFT}{\href{http://www.ccp5.ac.uk/software/}{http://www.ccp5.ac.uk/software/}}
\newcommand{\FORGE}{\href{http://ccpforge.cse.rl.ac.uk}{http://ccpforge.cse.rl.ac.uk}}
\newcommand{\vek}[1]{\mbox{$\underline{#1}$}}
\newcommand{\mat}[1]{\mbox{$\underline{\underline{\bf #1}}$}}
\newcommand{\para}[1]{\paragraph{#1}}
%\newcommand{\tilde}[1]{\mbox{$\~{#1}$}}

\makeindex


\begin{document}
\pagenumbering{alph} \setcounter{page}{0}
\addcontentsline{toc}{chapter}{\underline{The \D{} User Manual}}

\begin{titlepage}
\title{The \D{} User Manual}
\author{W. Smith, T.R. Forester and I.T. Todorov\\ \\
STFC  Daresbury Laboratory \\
Daresbury, Warrington WA4 4AD \\
Cheshire, UK}
\date{Version 1.10,~~January 2017}
\end{titlepage}

\maketitle
\clearpage

\pagenumbering{roman} \setcounter{page}{1}

\maketitle \addcontentsline{toc}{section}{About \D{}}
\input about_dlpoly.tex
\maketitle \addcontentsline{toc}{section}{Disclaimer}
\input disclaimer.tex
\maketitle \addcontentsline{toc}{section}{Acknowledgements}
\input acknowledgements.tex
\maketitle \addcontentsline{toc}{section}{Manual Notation}
\input manual_notation.tex

%pagestyle
\rhead{Contents}

\addcontentsline{toc}{chapter}{\underline{Contents}}
\tableofcontents
\clearpage
\addcontentsline{toc}{chapter}{\underline{List of Tables}}
\listoftables
\clearpage
\addcontentsline{toc}{chapter}{\underline{List of Figures}}
\listoffigures
\clearpage

\pagenumbering{arabic} \setcounter{page}{1}

%pagestyle
\rhead{Section \thesection}

\chapter{Introduction}
\label{intro}
\setcounter{equation}{0}
\newpage
\section*{Scope of Chapter}
This chapter describes the concept, design and directory structure
of \D{} and how to obtain a copy of the source code.
\newpage
\input introduction.tex

\chapter{Force Fields and Algorithms}
\label{field}
\setcounter{equation}{0}
\newpage
\section*{Scope of Chapter}
This chapter describes the interaction potentials and simulation
algorithms coded into \D{}.
\newpage
\input forfield.tex
\input integration.tex
\input repdata.tex
\newpage
\chapter{Construction and Execution}
\label{conex}
\setcounter{equation}{0}
\newpage
\section*{Scope of Chapter}
This chapter describes how to compile a working version of \D{}
and how to run it.
\newpage
\input construction.tex
\input compile.tex
\input ewald.tex
\input diagnos.tex
\newpage

\chapter{Data Files}
\label{files}
\setcounter{equation}{0}
\newpage
\section*{Scope of Chapter}
This chapter describes the standard input and output files for \D{},
examples of which are to be found in the {\em data} sub-directory.
\newpage
\input input.tex
\newpage
\input output.tex
\newpage

\chapter{Solvation}
\label{solvation}
\setcounter{equation}{0}
\newpage
\section*{Scope of Chapter}
This chapter describes the features within \D{} relevant to the subject of
solvation.  The main features are: decomposing the system configuration energy
into its molecular components; free energy calculations by thermodynamic
integration; and the calculation of solvent induced spectral shifts.  Some of
these features are sufficently general to have applications in other areas
besides solutions.
\newpage
\input solvation.tex
\newpage

\chapter{Hyperdynamics}
\label{hyperdynamics}
\setcounter{equation}{0}
\newpage
\section*{Scope of Chapter}
This chapter describes the facilities within \D{} for performing
accelerated dynamics (or hyperdynamics) using the {\em Bias Potential
Dynamics} and {\em Temperature Accelerated Dynamics} methods.
\newpage
\input hyper.tex
\newpage

\chapter{Metadynamics}
\label{metadynamics}
\setcounter{equation}{0}
\newpage
\section*{Scope of Chapter}
This chapter describes the facilities within \D{} for studying the 
thermodynamics of phase transitions using the method of metadynamics.
\newpage
\input meta.tex
\newpage

\chapter{Path Integral Molecular Dynamics}
\label{pimd}
\setcounter{equation}{0}
\newpage
\section*{Scope of Chapter}
This chapter describes the implementation of Path Integral Molecular
Dynamics (PIMD) in \D{}, which permits the calculation of quantum
corrected thermodynamic and structural properties for atomistic
systems.
\newpage
\input pimd.tex
\newpage

\begin{comment}
\chapter{The Java GUI}
\label{javagui}
\setcounter{equation}{0}
\newpage
\section*{Scope of Chapter}
This chapter describes the \D{} Java Graphical User Interface which
offers utilities for running \D{}, constructing example input files
and performing basic analysis of \D{} output.
\newpage
\input javagui.tex
\newpage
\end{comment}

\chapter{Example Simulations}
\label{data}
\setcounter{equation}{0}
\newpage
\section*{Scope of Chapter}
This chapter describes the standard test cases for \D{}, the input
and output files for which are in the {\em data} sub-directory.
\newpage
\input examples.tex

\chapter{Utilities}
\label{utility}
\setcounter{equation}{0}
\newpage
\section*{Scope of Chapter}
This chapter describes the more important utility programs and
subroutines of \D{}, found in the sub-directory {\em utility}. 

\section{Miscellaneous Utilities}
\input macros.tex

\clearpage
\addcontentsline{toc}{chapter}{Bibliography}
\bibliographystyle{dl_poly}
\bibliography{dl_poly}

\clearpage
\appendix
\addcontentsline{toc}{chapter}{\underline{Appendices}}


\chapter{The \D{} Makefile}
\input makefile.tex
\chapter{Periodic Boundary Conditions in \D{}}
\input boundaries.tex
\chapter{Error Messages and User Action}
\input messages.tex
\chapter{Subroutine Locations}
\input locations.tex

\clearpage
\addcontentsline{toc}{chapter}{Index}
\printindex
\end{document}
