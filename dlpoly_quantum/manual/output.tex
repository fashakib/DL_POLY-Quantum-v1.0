
\section{The OUTPUT Files}
\label{outputfiles}

\D{} produces up to eight output files: HISTORY, OUTPUT, REVCON, REVIVE,
RDFDAT, ZDNDAT, STATIS and CFGMIN. These respectively contain: a dump file of
atomic coordinates, velocities and forces; a summary of the
simulation; the restart configuration; statistics accumulators; radial
distribution data, Z-density data, a statistical history, and the
configuration with the lowest configurational energy. Some of these
files are optional and appear only when certain options are used.

{\bf Note:} In addition to the files described in this chapter, users
of the hyperdynamics features of \D{} should see Chapter
\ref{hyperdynamics}, where additional files specific to that purpose
are described. Similarly, the output files specific to the solvation
features of \D{} are described in Chapter \ref{solvation}. The output
files specific to the path integral (PIMD) option are described in
Chapter \ref{pimd}.

\subsection{The HISTORY File}
\label{historyfile}
\index{HISTORY file}

The HISTORY file is the dump file of atomic coordinates, velocities
and forces. Its principal use is for off-line analysis. The file is
written by the subroutines {\sc traject} or {\sc traject\_u}. The
control variables for this file are {\tt ltraj, nstraj, istraj} and
{\tt keytrj} which are created internally, based on information read
from the {\bf traj} directive in the CONTROL file (see above). The
HISTORY file will be created only if the directive {\bf traj} appears
in the CONTROL file.  Note that the HISTORY file can be written in
either a formatted or unformatted version. We describe each of these
separately below. If you want your HISTORY data to have maximum numerical
precision, you should use the unformatted version.

The HISTORY file can become {\em very} large, especially if it is
formatted. For serious simulation work it is recommended that the file
be written to a scratch disk capable of accommodating a large data
file.  Alternatively the file may be written as unformatted (see below),
which has the additional advantage of speed. However, writing an
unformatted file has the disadvantage that the file may not be
readable except by the machine on which it was created. This is
particularly important if graphical processing of the data is
required.

\subsubsection{The Formatted HISTORY File}
\index{HISTORY file! formatted}

The formatted HISTORY file is written by the subroutine {\sc traject}
and  has the following structure.
\begin{tabbing}
X\=XXXXXXXX\=XXXXXXXXXXXXXXXX\=\kill
{\bf record 1} (a80)\\
\> {\tt header} \> a80 \> file header \\
{\bf record 2} (3i10)\\
\> {\tt keytrj} \> integer \> trajectory key (see table \ref{KEYTRJ})\\
\> {\tt imcon} \> integer \> periodic boundary key (see table \ref{IMCON})\\
\> {\tt natms} \> integer \> number of atoms in simulation cell\\
\end{tabbing}

For timesteps greater than {\tt nstraj} the HISTORY file is appended at
intervals specified by the {\bf traj} directive in the CONTROL file,
with the following information for each configuration:

\begin{tabbing}
X\=XXXXXXXXXXXXX\=XXXXXXXXXXXXXX\=\kill
{\bf record i} (a8,4i10,f12.6)\\
\> {\tt timestep} \> a8 \> the character string ``timestep''\\
\> {\tt nstep} \>  integer \> the current time-step\\
\> {\tt natms} \> integer \> number of atoms in configuration\\
\> {\tt keytrj} \> integer \> trajectory key (again) \\
\> {\tt imcon} \> integer \> periodic boundary key (again) \\
\> {\tt tstep} \> real \> integration timestep \\
{\bf record ii} (3g12.4) for {\tt imcon} $>$ 0\\
\> {\tt cell(1)} \> real \> x component of $a$ cell vector\\
\> {\tt cell(2)} \> real \> y component of $a$ cell vector\\
\> {\tt cell(3)} \> real \> z component of $a$ cell vector\\
{\bf record iii} (3g12.4) for {\tt imcon} $>$ 0\\
\> {\tt cell(4)} \> real \> x component of $b$ cell vector\\
\> {\tt cell(5)} \> real \> y component of $b$ cell vector\\
\> {\tt cell(6)} \> real \> z component of $b$ cell vector\\
{\bf record iv}  (3g12.4) for {\tt imcon} $>$ 0\\
\> {\tt cell(7)} \> real \> x component of $c$ cell vector\\
\> {\tt cell(8)} \> real \> y component of $c$ cell vector\\
\> {\tt cell(9)} \> real \> z component of $c$ cell vector\\
\end{tabbing}
This is followed by the configuration for the current timestep. i.e.
for each atom in the system the following data are included:
\begin{tabbing}
X\=XXXXXXXX\=XXXXXXXXXXXXXXXX\=\kill
{\bf record a} (a8,i10,2f12.6) \\
\> {\tt atmnam} \> a8 \> atomic label\\
\> {\tt iatm} \> i10 \>  atom index\\
\> {\tt weight} \> f12.6 \> atomic mass (a.m.u.)\\
\> {\tt  charge} \> f12.6 \> atomic charge (e) \\
{\bf record b} (3e12.4)\\
\> {\tt xxx} \> real \> x coordinate\\
\> {\tt yyy} \> real \> y coordinate\\
\> {\tt zzz} \> real \> z coordinate\\
{\bf record c} (3e12.4) only for {\tt keytrj} $>$ 0\\
\> {\tt vxx} \> real \> x component of velocity\\
\> {\tt vyy} \> real \> y component of velocity\\
\> {\tt vzz} \> real \> z component of velocity\\
{\bf record d} (3e12.4) only for {\tt keytrj} $>$ 1\\
\> {\tt fxx} \> real \> x component of force\\
\> {\tt fyy} \> real \> y component of force\\
\> {\tt fzz} \> real \> z component of force\\
\end{tabbing}
Thus the data for each atom is a minimum of two records and a maximum
of 4.

\subsubsection{The Unformatted HISTORY File}
\index{HISTORY file! unformatted}

The unformatted HISTORY file is written by the subroutine {\sc
traject\_u} and has the following structure:

\begin{tabbing}
X\=XXXXXXXXXXXXXXXX\=\kill
{\bf record 1}\\
\> {\tt header} \> configuration name (character*80)\\
{\bf record 2}\\
\> {\tt natms} \> number of atoms in the configuration (real*8)\\
{\bf record 3}\\
\> {\tt atname(1,...,natms)} \> atom names or symbols (character*8)\\
{\bf record 4}\\
\> {\tt weight(1,...,natms)} \> atomic masses (real*8)\\
{\bf record 5}\\
\> {\tt charge(1,...,natms)} \> atomic charges (real*8)\\
\end{tabbing}

For time-steps greater than {\tt nstraj}, the HISTORY file is appended, at
intervals specified by the {\bf traj} directive in the CONTROL file,
with the following information:

\begin{tabbing}
X\=XXXXXXXXXXXXXXXX\=\kill
{\bf record i}\\
\> {\tt nstep} \> the current time-step (real*8)\\
\> {\tt natms} \> number of atoms in configuration (real*8)\\
\> {\tt keytrj} \> trajectory key (real*8)\\
\> {\tt imcon} \> image convention key (real*8)\\
\> {\tt tstep} \> integration timestep (real*8) \\
{\bf record ii} for {\tt imcon} $>$ 0\\
\> {\tt cell(1,...,9)} \> $a,~b$ and $c$  cell vectors (real*8)\\
{\bf record iii}\\
\> {\tt xxx(1,...,natms)} \> atomic x-coordinates (real*8)\\
{\bf record iv}\\
\> {\tt yyy(1,...,natms)} \> atomic y-coordinates (real*8)\\
{\bf record v}\\
\> {\tt zzz(1,...,natms)} \> atomic z-coordinates (real*8)\\
{\bf record vi} only for {\tt keytrj}$>0$\\
\> {\tt vxx(1,...,natms)} \> atomic velocities x-component (real*8)\\
{\bf record vii} only for {\tt keytrj}$>0$\\
\> {\tt vyy(1,...,natms)} \> atomic velocities y-component (real*8)\\
{\bf record viii} only for {\tt keytrj}$>0$\\
\> {\tt vzz(1,...,natms)} \> atomic velocities z-component (real*8)\\
{\bf record ix} only for {\tt keytrj}$>1$\\
\> {\tt fxx(1,...,natms)} \> atomic forces x-component (real*8)\\
{\bf record x} only for {\tt keytrj}$>1$\\
\> {\tt fyy(1,...,natms)} \> atomic forces y-component (real*8)\\
{\bf record xi} only for {\tt keytrj}$>1$\\
\> {\tt fzz(1,...,natms)} \> atomic forces z-component (real*8)\\
\end{tabbing}
Note the implied conversion of integer variables to real on record i.

\subsection{The OUTPUT File}
\label{outputfile}
\index{OUTPUT file}

The job output consists of 7 sections: Header; Simulation control
specifications; Force field specification; Summary of the initial
configuration; Simulation progress; Summary of statistical data;
Sample of the final configuration; and Radial distribution functions.
These sections are written by different subroutines at various stages
of a job. Creation of the OUTPUT file {\em always} results from
running \D{}. It is meant to be a human readable file, destined
for hardcopy output.

\subsubsection{Header}

Gives the \D{} version number, the number of processors used and a
title for the job as given in the header line of the input file
CONTROL.  This part of the file is written from the subroutines
{\sc dlpoly} and  {\sc simdef}

\subsubsection{Simulation Control Specifications}

Echoes the input from the CONTROL file. Some variables may be reset if
illegal values were specified in the CONTROL file.  This part of
the file is written from the subroutine {\sc simdef}.

\subsubsection{Force Field Specification}

Echoes the FIELD file. A warning line will be printed if the system is
not electrically neutral. This warning will appear immediately before
the non-bonded short-range potential specifications.  This part of the
file is written from the subroutine {\sc sysdef}. 

\subsubsection{Summary of the Initial Configuration}

 This part of the file is written from the subroutine {\sc sysgen}. It
states the periodic boundary specification, the cell vectors and
volume (if appropriate) and the initial configuration of (a maximum
of) 20 atoms in the system. The configuration information given is
based on the value of {\tt levcfg} in the CONFIG file. If {\tt levcfg}
is 0 (or 1) positions (and velocities) of the 20 atoms are listed. If
{\tt levcfg} is 2 forces are also written out.

For periodic systems this is followed by the long range corrections to
the energy and pressure.

\subsubsection{Simulation Progress}

 This part of the file is written by the \D{} root segment {\sc dlpoly}.
The header line is printed at the top of each page as: 

{\scriptsize\begin{verbatim}

--------------------------------------------------------------------------------------------------

    step   eng_tot  temp_tot   eng_cfg   eng_vdw   eng_cou   eng_bnd   eng_ang   eng_dih   eng_tet
 time       eng_pv  temp_rot   vir_cfg   vir_vdw   vir_cou   vir_bnd   vir_ang   vir_con   vir_tet
 cpu time   volume  temp_shl   eng_shl   vir_shl     alpha      beta   gamma   vir_pmf     press

--------------------------------------------------------------------------------------------------
\end{verbatim}}

The labels refer to :
\begin{tabbing}
X\=XXXXXXXXXX\=XXXXXXXXXXXXXXXXXXXXXXXXXXXXXXXXXX\=\kill
{\bf line 1}\\
\>step \> MD step number\\
\> {\tt eng\_tot} \> total internal energy of the system\\
\> {\tt temp\_tot} \> system temperature\\
\> {\tt eng\_cfg} \> configurational energy of the system\\
\> {\tt eng\_vdw} \> configurational energy due to short-range
potentials\\
\> {\tt eng\_cou} \> configurational energy due to electrostatic
potential\index{potential!electrostatic} \\
\> {\tt eng\_bnd} \> configurational energy due to chemical bond\index{potential!bond}
potentials\\
\> {\tt eng\_ang} \> configurational energy due to valence angle\index{potential!valence angle}
and three-body\index{potential!three-body} potentials\\
\> {\tt eng\_dih} \> configurational energy due to dihedral\index{potential!dihedral}
inversion and four-body\index{potential!four-body} potentials\\
\> {\tt eng\_tet} \> configurational energy due to tethering
potentials\index{potential!tethered}\\
{\bf line 2}\\
\> {\tt time} \> elapsed simulation time (fs,ps,ns) since the beginning
of the job\\
\> {\tt eng\_pv} \> enthalpy of system\\
\> {\tt temp\_rot} \> rotational temperature\\
\> {\tt vir\_cfg} \> total configurational contribution to the
virial\\
\> {\tt vir\_vdw} \> short range potential contribution to the
virial\\
\> {\tt vir\_cou} \> electrostatic potential\index{potential!electrostatic} contribution to the
virial\\
\> {\tt vir\_bnd} \> chemical bond contribution to the virial\\
\> {\tt vir\_ang} \> angular and three body\index{potential!three-body} potentials contribution to
the virial\\
\> {\tt vir\_con} \> constraint bond\index{constraints!bond} contribution to the virial\\
\> {\tt vir\_tet} \> tethering potential\index{potential!tethered} contribution to the virial\\
\end{tabbing}

{\bf Note:} The total internal energy of the system (variable {\tt
tot\_energy}) includes all contributions to the energy (including
system extensions due to thermostats etc.) It is nominally the {\em
conserved variable} of the system, and is not to be confused with
conventional system energy, which is a sum of the kinetic and
configuration energies.

In cases where the PIMD option is active, and additional data line
appears :
\begin{tabbing}
X\=XXXXXXXXXX\=XXXXXXXXXXXXXXXXXXXXXXXXXXXXXXXXXX\=\kill
{\bf line 4}\\
\> {\tt eng\_qpi} \> energy of the quantum system (standard energy estimator)\\
\> {\tt eng\_qvr} \> energy of the quantum system (virial energy estimator)\\
\> {\tt eng\_rng} \> energy contribution due to the quantum rings\\
\> {\tt vir\_rng} \> virial contribution due to the quantum rings\\
\> {\tt qms\_rgr} \> mean-square radius of gyration of the quantum rings\\
\> {\tt qms\_bnd } \> mean-square bondlength of the quantum rings\\
\> {\tt eng\_the } \> total energy of the quantum ring thermostats\\
\end{tabbing}

The interval for printing out these data is determined by the
directive {\bf print} in the CONTROL file.  At each time-step that
printout is requested the instantaneous values of the above
statistical variables are given in the appropriate columns.
Immediately below these three lines of output the rolling averages of
the same variables are also given. The maximum number of
time-steps\index{algorithm!multiple timestep} used to calculate the
rolling averages is determined by the parameter {\tt mxstak} defined
in the {\sc setup\_module./f} file.
The working number of time-steps for rolling averages
is controlled by the directive {\bf stack} in file CONTROL (see
above).  The default value is {\tt mxstak}.

\paragraph*{Energy Units:}

The energy unit for the data appearing in the OUTPUT is defined by the
{\bf units} directive appearing in the CONTROL file. 

\paragraph*{Pressure units:}

The unit of pressure\index{units!pressure} is {\em k~atm},
irrespective of what energy unit is chosen.

\subsubsection{Summary of Statistical Data}

This portion of the OUTPUT file is written from the subroutine {\sc
result}.  The number of time-steps used in the collection of
statistics is given.  Then the averages over the production portion of
the run are given for the variables described in the previous section.
The root mean square variation in these variables follow on the next
two lines. The energy\index{units!DL\_POLY} and pressure
units\index{units!pressure} are as for the preceeding section.

Also provided in this section is an estimate of the diffusion
coefficient for the different species in the simulation, which is
determined from a {\em single time origin} and is therefore very
approximate. Accurate determinations of the diffusion coefficients can
be obtained using the {\sc msd} utility program, which processes the
HISTORY file (see chapter \ref{utility}).

If an NPT or N$\mat{\sigma}$T simulation is performed the OUTPUT file
also provides the mean stress (pressure) tensor and mean simulation
cell vectors.

Since version 1.10 of \D{} this section also includes a record of the
even gaussian moments of the momentum (up to moment 10) expressed the
ratio of the simulated and theoretical values (see equation
\ref{nmom}). These are intended to show the degree to which the
simulation distribution function is properly represented as a Gaussian
function i.e. that it complies with Boltzmann's distribution. It is
importnat to note however, that the underlying theory of this is valid
only for atoms undergoing unrestricted motion; atoms that are linked
by constraint bonds or are embedded in rigid bodies will not behave in
the right way for this test to be valid.

\subsubsection{Sample of Final Configuration}

The positions, velocities and forces of the 20 atoms used for the
sample of the initial configuration (see above) are given. This is
written by the subroutine {\sc result}.
 
\subsubsection{Radial Distribution Functions}

If both calculation and printing of radial distribution functions have
been requested (by selecting directives {\bf rdf} and {\bf print rdf}
in the CONTROL file) radial distribution functions are printed out. 
This is written from the subroutine {\sc
rdf1}. First the number of time-steps used for the collection of the
histograms is stated.  Then each function is given in turn. For each
function a header line states the atom types (`a' and `b') represented
by the function. Then $r,~g(r)$ and $n(r)$ are given in tabular form.
Output is given from 2 entries before the first non-zero entry in the
$g(r)$ histogram.  $n(r)$ is the average number of atoms of type `b'
within a sphere of radius $r$ around an atom of type `a'.
Note that a readable version of these data is provided by the 
RDFDAT file (below). 

\subsubsection{ Z Density Profile}
If both calculation and printing of Z density profiles has been
requested (by selecting directives {\bf zden} and {\bf print rdf} in
the CONTROL file Z density profiles are printed out as the last part
of the OUTPUT file.  This is written by the subroutine {\sc zden1}.
First the number of time-steps used for the collection of the
histograms is stated.  Then each function is given in turn. For each
function a header line states the atom type represented by the
function.  Then $z,~\rho(z)$ and $n(z)$ are given in tabular form.
Output is given from $Z = [-L/2,L/2]$ where L is the length of the MD
cell in the Z direction and $\rho(z)$ is the mean number density.
$n(z)$ is the running integral from $-L/2$ to $z$ of $({\rm
xy~cell~area})\rho(s) ds$.
Note that a readable version of these data is provided by the 
ZDNDAT file (below). 

\subsection{The REVCON File}
\label{revconfile}
\index{REVCON file}

This file is formatted and written by the subroutine {\sc revive}.
REVCON is the restart configuration file.  The file is written every
{\tt ndump} time steps in case of a system crash during execution and
at the termination of the job. A successful run of \D{} will always
produce a REVCON file, but a failed job may not produce the file if an
insufficient number of timesteps have elapsed. {\tt ndump} is a
parameter defined in the {\sc setup\_module.f} file found in the {\em
source} directory of \D{}.  Changing {\tt
ndump} necessitates recompiling \D{}.  

REVCON is identical in format to the CONFIG input file (see section
\ref{configfile}).

REVCON should be renamed CONFIG to continue a simulation from one job
to the next.  This is done for you by the {\sl copy} macro supplied in
the {\em execute} directory of \D{}.

\subsection{The CFGMIN File}
\label{cfgminfile}
\index{CFGMIN file}

The CFGMIN file only appears if the user has selected the programmed
minimisation option (directive {\bf minim} in the CONTROL file). Its
contents have the same format as the CONFIG file (see section
\ref{configfile}), but contains only atomic position data and will
never contain either velocity or force data (i.e. parameter {\tt
levcfg} is always zero).  In addition, two extra numbers appear on the
end of the second line of the file:
\begin{enumerate}
\item an integer indicating the number of minimisation cycles required
to obtain the structure (format I10);
\item the configuration energy of the final structure expressed in 
DL\_POLY units \ref{units} (format F20).
\end{enumerate}

\subsection{The REVIVE File}
\label{revivefile}
\index{REVIVE file}

This file is unformatted and written by the subroutine {\sc revive}.
It contains the accumulated statistical data. It is updated whenever
the file REVCON is updated (see previous section). REVIVE should be
renamed REVOLD to continue a simulation from one job to
the next.  This is done by the {\sl copy} macro supplied in the
{\em execute} directory of \D{}. In addition, to continue a 
simulation from a previous job the {\bf restart} keyword must be
included in the CONTROL file.

The format of the REVIVE file is identical to the REVOLD file
described in section \ref{revoldfile}.

\subsection{The RDFDAT File}
\label{rdffile}
\index{RDFDAT file}

This is a formatted file containing {em Radial Distribution Function}
(RDF) data. Its contents are as follows:

\begin{tabbing}
X\=XXXXXXXX\=XXXXXXXXXXXX\=XXXXXXXXXX\=\kill
{\bf record 1}\\
\> {\tt cfgname} \> character (A80)\> configuration name\\
{\bf record 2}\\
\> {\tt ntpvdw} \> integer (i10) \> number of RDFs in file\\
\> {\tt mxrdf} \> integer (i10) \> number of data points in each RDF\\
\end{tabbing}

There follow the data for each individual RDF i.e. {\em ntpvdw}
times. The data supplied are as follows:

\begin{tabbing}
X\=XXXXXXXX\=XXXXXXXXXXXX\=XXXXXXXXXX\=\kill
{\bf first record}\\
\> {\tt atname 1} \> character (A8)\> first atom name\\
\> {\tt atname 2} \> character (A8)\> second atom name\\
{\bf following records} ({\em mxrdf} records)\\
\> {\tt radius} \> real (e14) \> interatomic distance (A)\\
\> {\tt g(r)} \> real (e14) \> RDF at given radius.\\
\end{tabbing}
Note the RDFDAT file is optional and appears when the {\bf print rdf}
option is specified in the CONTROL file.

\subsection{The ZDNDAT File}
\label{zdnfile}
\index{ZDNDAT file}

This is a formatted file containing the Z-density data. Its contents
are as follows:

\begin{tabbing}
X\=XXXXXXXX\=XXXXXXXXXXXX\=XXXXXXXXXX\=\kill
{\bf record 1}\\
\> {\tt cfgname} \> character (A80)\> configuration name\\
{\bf record 2}\\
\> {\tt mxrdf} \> integer (i10) \> number of data points in the
Z-density function\\
{\bf following records} ({\em mxrdf} records)\\
\> {\tt z} \> real (e14) \> distance in z direction (A)\\
\> $\rho(z)$ \> real (e14) \> Z-density at given height {\tt z}\\
\end{tabbing}

Note the ZDNDAT file is optional and appears when the {\bf print rdf}
option is specified in the CONTROL file.

\subsection{The STATIS File}
\label{statisfile}
\index{STATIS file}

The file is formatted, with integers as ``i10'' and reals as
``e14.6''.  It is written by the subroutine {\sc static}. It consists
of two header records followed by many data records of statistical
data.
\begin{tabbing}
X\=XXXXXXXX\=XXXXXXXXXXXX\=XXXXXXXXXX\=\kill
{\bf record 1}\\
\> {\tt cfgname} \> character \> configuration name\\
{\bf record 2}\\
\> {\tt string} \> character \> energy units\\
\end{tabbing}

{\bf Data records}\\ Subsequent lines contain the instantaneous values
of statistical variables dumped from the array {\tt stpval}. A
specified number of
entries of {\tt stpval} are written in the format ``(1p,5e14.6)''. 
The number of array elements required (determined by the parameter
{\tt mxnstk} in
the {\sc dl\_params.inc} file) is
\begin{eqnarray}
{\tt mxnstk} \ge& 27  + {\tt ntpatm} (\rm
number~of~unique~atomic~sites) \nonumber\\
                & +  9 (\rm if~stress~tensor~calculated) \nonumber\\
                & +  9 (\rm if~constant~pressure~simulation~requested)
\nonumber
\end{eqnarray}
The STATIS file is appended at intervals determined by the
{\bf stats} directive in the CONTROL file. The energy
unit is as specified in the CONTROL file with the the {\bf units}
directive, and are compatible with the data appearing in the OUTPUT
file.  The contents of the appended information is:
\begin{tabbing}
X\=XXXXXXXXXXXX\=XXXXXXXXXXXX\=XXXXXXXXXXXX\=\kill
{\bf record i}\\
\> {\tt nstep} \> integer \> current MD time-step\\
\> {\tt  time }\>  real  \> elapsed simulation time = {\tt
nstep}$\times \Delta t$\\
\> {\tt nument} \> integer \> number of array elements to follow\\
{\bf record ii} {\tt stpval}(1) -- {\tt stpval}(5)\\
\> {\tt engcns} \> real \> total extended system energy \\
\> \> \> (i.e. the conserved quantity)\\
\> {\tt temp} \> real \> system temperature\\
\> {\tt engcfg} \> real \> configurational energy\\
\> {\tt engsrp} \> real \> VdW/metal/Tersoff energy\\
\> {\tt engcpe} \> real \> electrostatic energy\\
{\bf record iii}  {\tt stpval}(6) -- {\tt stpval}(10)\\
\> {\tt engbnd} \> real \> chemical bond energy\\
\> {\tt engang} \> real \> valence angle/3-body potential energy\\
\> {\tt engdih} \> real \> dihedral/inversion/four body energy\\
\>{\tt engtet} \> real \> tethering energy\\
\>{\tt enthal} \> real \> enthalpy (total energy + PV)\\
{\bf record iv} {\tt stpval}(11) -- {\tt stpval}(15)\\
\> {\tt tmprot} \> real \> rotational temperature\\
\> {\tt vir} \> real \> total virial\\
\> {\tt virsrp} \> real \> VdW/metal/Tersoff virial\\
\> {\tt vircpe} \> real \> electrostatic virial\\
\> {\tt virbnd} \> real \> bond virial\\
{\bf record v} {\tt stpval}(16) -{\tt stpval}(20)\\
\> {\tt virang} \> real \> valence angle/3-body virial\\
\> {\tt vircon} \> real \> constraint virial\\
\> {\tt virtet} \> real \> tethering virial\\
\> {\tt volume} \> real \> volume\\
\> {\tt tmpshl} \> real \> core-shell temperature\\
{\bf record vi} {\tt stpval}(21) -{\tt stpval}(25)\\
\> {\tt engshl} \> real \> core-shell potential energy\\
\> {\tt virshl} \> real \> core-shell virial\\
\> {\tt alpha } \> real \> MD cell angle $\alpha$\\
\> {\tt beta } \> real \> MD cell angle $\beta$\\
\> {\tt gamma } \> real \> MD cell angle $\gamma$\\
{\bf record vii} {\tt stpval}(26) -{\tt stpval}(27)\\
\> {\tt virpmf} \> real \> Potential of Mean Force virial\\
\> {\tt press} \> real \> pressure\\
{\bf the next {\tt ntpatm} entries}\\
\> {\tt amsd(1)} \> real \> mean squared displacement of first atom
types\\
\> {\tt amsd(2)} \> real \> mean squared displacement of second atom
types\\
\> {\tt ...} \> ... \> ... \\
\> {\tt amsd(ntpatm)} \> real \> mean squared displacement of last
atom types\\
{\bf the next 9 entries - {\em if} the stress tensor is calculated}\\
\> {\tt stress(1)} \> real \> xx component of stress tensor\\
\> {\tt stress(2)} \> real \> xy component of stress tensor\\
\> {\tt stress(3)} \> real \> xz component of stress tensor\\
\> {\tt stress(4)} \> real \> yx component of stress tensor\\
\> {\tt ...} \> real \> ... \\
\> {\tt stress(9)} \> real \> zz component of stress tensor\\
{\bf the next 9 entries - {\em if} a NPT simulation is undertaken}\\
\> {\tt cell(1)} \> real \> x component of $a$ cell vector\\
\> {\tt cell(2)} \> real \> y component of $a$ cell vector\\
\> {\tt cell(3)} \> real \> z component of $a$ cell vector\\
\> {\tt cell(4)} \> real \> x component of $b$ cell vector\\
\> {\tt ...} \> real \> ... \\
\> {\tt cell(9)} \> real \> z component of $c$ cell vector\\
\end{tabbing}

