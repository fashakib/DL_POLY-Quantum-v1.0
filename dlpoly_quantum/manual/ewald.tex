
\subsection{Choosing Ewald Sum Variables}
\label{ewaldoptim}

\subsubsection{Ewald sum and SPME}

This section outlines how to optimise the accuracy of the Ewald sum
parameters for a given simulation. In what follows the directive {\bf
spme} may be used anywhere in place of the directive {\bf ewald} if
the user wishes to use the Smoothed Particle Mesh Ewald
\index{Ewald!SPME} method.

As a guide\index{Ewald!optimisation} to beginners \D{} will calculate
reasonable parameters if the {\bf ewald precision} directive is used
in the CONTROL file (see section \ref{controlfile}). A relative error
(see below) of 10$^{-6}$ is normally sufficient so the directive

\vskip 1em
\noindent
{\bf ewald precision 1d-6} 


\vskip 1em
\noindent
will cause \D{} to evaluate its best guess at the Ewald parameters 
$\alpha$, {\tt kmax1}, {\tt kmax2} and {\tt kmax3}. 
(The user should note that this represents an {\em estimate}, and there
are sometimes circumstances where the estimate can be improved
upon. This is especially the case when the system contains a strong
directional anisotropy, such as a surface.) These four parameters
may also be set explicitly by the {\bf ewald sum } directive in the 
CONTROL file. For example the directive

\vskip 1em
\noindent
{\bf ewald sum 0.35 6 6 8}

\vskip 1em
\noindent
would set $\alpha= 0.35$ \AA$^{-1}$, {\tt kmax1} = 6, {\tt kmax2 = 6}
and {\tt kmax3 } = 8. The quickest check on the accuracy of the Ewald
sum\index{Ewald!summation} is to compare the Coulombic energy ($U$) and the coulombic virial
($\cal W$) in a short simulation.  Adherence to the relationship $U =
-{\cal W}$ shows the extent to which the Ewald sum\index{Ewald!summation} is correctly
converged. These variables can be found under the columns headed {\tt
eng\_cou} and {\tt vir\_cou} in the OUTPUT file (see section
\ref{outputfile}). 

The remainder of this section explains the meanings of these
parameters and how they can be chosen.  The Ewald
sum\index{Ewald!summation} can only be used in a three dimensional
periodic system.  There are three variables that control the accuracy:
$\alpha$, the Ewald convergence parameter; $r_{\rm cut}$ the real
space forces cutoff; and the {\tt kmax1,2,3} integers \footnote{{\bf
Important note:} For the SPME method the values of {\tt kmax1,2,3}
should be double those obtained in this prescription, since they
specify the sides of a cube, not a radius of convergence.}  that
effectively define the range of the reciprocal space sum (one integer
for each of the three axis directions).  These variables are not
independent, and it is usual to regard one of them as pre-determined
and adjust the other two accordingly. In this treatment we assume that
$r_{\rm cut}$ (defined by the {\bf cutoff} directive in the CONTROL
file) is fixed for the given system.

The Ewald sum splits the (electrostatic) sum for the infinite,
periodic, system into a damped real space sum and a reciprocal space
sum. The rate of convergence of both sums is governed by $\alpha$.
Evaluation of the real space sum is truncated at $r=r_{\rm cut}$ so it
is important that $\alpha$ be chosen so that contributions to the real
space sum are negligible for terms with $r>r_{\rm cut}$.  The relative
error ($\epsilon$) in the real space sum truncated at $r_{\rm cut}$ is
given approximately by\index{Ewald!optimisation}
\begin{equation}
\epsilon \approx {\rm erfc}(\alpha r_{\rm cut})/r_{\rm cut} 
\approx \exp[-(\alpha.r_{\rm cut})^2]/r_{\rm cut} \label{relerr}
\end{equation}

The recommended value for $\alpha$ is 3.2/$r_{\rm cut}$ or greater
(too large a value will make the reciprocal space sum very slowly
convergent). This gives a relative error in the energy of no greater
than $\epsilon = 4\times 10^{-5}$ in the real space sum. When using
the directive {\bf ewald precision} \D{} makes use of a more sophisticated
approximation:
\begin{equation}
{\rm erfc}(x) \approx 0.56 \exp(-x^2)/x
\end{equation}
to solve recursively for $\alpha$, using equation \ref{relerr} to give
the first guess.

The relative error in the reciprocal space term is approximately
\begin{equation}
\epsilon \approx \exp(- k_{max}^2/4\alpha^2)/k_{max}^2
\end{equation}
where
\begin{equation}
k_{max} = \frac{2\pi}{L}~{\tt kmax}
\end{equation}
is the largest $k$-vector considered in reciprocal space, $L$ is the
width of the cell in the specified direction and {\tt kmax} is an integer. 

For a relative error of $4\times 10^{-5}$ this means using $k_{max}
\approx 6.2 \alpha$.  {\tt kmax} is then
\begin{equation}
{\tt kmax} > 3.2~L/r_{\rm cut}
\end {equation}

In a cubic system, $r_{\rm cut}~=~L/2$ implies ${\tt kmax}~=~7$.  In
practice the above equation slightly over estimates the value of {\tt
kmax} required, so optimal values need to be found experimentally.  In
the above example {\tt kmax}~=~5 or 6 would be adequate.

If your simulation cell is a truncated octahedron or a rhombic
dodecahedron then the estimates for the {\tt kmax} need to be
multiplied by $2^{1/3}$. This arises because twice the normal number
of $k$-vectors are required (half of which are redundant by symmetry)
for these boundary contributions \cite{smith-93b}.

If you wish to set the Ewald parameters manually (via the {\bf ewald
sum} or {\em spme sum} directives) the recommended approach is as follows\index{Ewald!optimisation}. Preselect the
value of $r_{\rm cut}$, choose a working a value of $\alpha$ of about
$3.2/r_{\rm cut}$ and a large value for the {\tt kmax} (say 10 10 10
or more).  Then do a series of ten or so {\em single} step simulations
with your initial configuration and with $\alpha$ ranging over the
value you have chosen plus and minus 20\%. Plot the Coulombic energy
(and $-{\cal W}$) versus $\alpha$. If the Ewald sum\index{Ewald!summation} is correctly
converged you will see a plateau in the plot.  Divergence from the
plateau at small $\alpha$ is due to non-convergence in the real space
sum. Divergence from the plateau at large $\alpha$ is due to
non-convergence of the reciprocal space sum.  Redo the series of
calculations using smaller {\tt kmax} values. The optimum values for
{\tt kmax} are the smallest values that reproduce the correct
Coulombic energy (the plateau value) and virial at the value of
$\alpha$ to be used in the simulation.

Note that one needs to specify the three integers ({\tt kmax1, kmax2,
kmax3}) referring to the three spatial directions, to ensure the
reciprocal space sum is equally accurate in all directions. The values
of {\tt kmax1}, {\tt kmax2} and {\tt kmax3} must be commensurate with
the cell geometry to ensure the same minimum wavelength is used in all
directions.  For a cubic cell set {\tt kmax1} = {\tt kmax2} = {\tt
kmax3}.  However, for example, in a cell with dimensions $2A = 2B = C$
(ie. a tetragonal cell, longer in the c direction than the a and b
directions) use 2{\tt kmax}1 = 2{\tt kmax}2 = ({\tt kmax}3).

If the values for the {\tt kmax} used are too small, the Ewald sum\index{Ewald!summation} will
produce spurious results. If values that are too large are used, the
results will be correct but the calculation will consume unnecessary
amounts of {\em cpu} time. The amount of {\em cpu} time increases with
${\tt kmax1}\times{\tt kmax2} \times {\tt kmax3}$.

\subsubsection{Hautman Klein Ewald Optimisation}

Setting the HKE \index{Ewald!Hautman Klein} parameters can also be
achieved rather simply, by the use of a {\bf hke precision}
directive in the CONTROL file e.g.

\vskip 1em
\noindent
{\bf hke precision 1d-6 1 1} 


\vskip 1em
\noindent
which specifies the required accuracy of the HKE convergence
functions, plus two additional integers; the first specifying the
order of the HKE expansion ({\tt nhko}) and the second the maximum
lattice parameter ({\tt nlatt}). \D{} will permit values of {\tt nhko}
from 1-3, meaning the HKE Taylor series expansion may range from
zeroth to third order. Also {\tt nlatt} may range from 1-2, meaning
that (1) the nearest neighbour, and (2) and next nearest neighbour,
cells are explicitly treated in the real space part of the Ewald
sum. Increasing either of these parameters will increase the accuracy,
but also substantially increase the cpu time of a simulation. The
recommended value for both these parameters is 1 and if {\em both} these
integers are left out, the default values will be adopted.

As with the standard Ewald and SPME methods, the user may set alternative
control parameters with the CONTROL file {\bf hke sum} directive e.g.

\vskip 1em
\noindent
{\bf hke sum 0.05 6 6 1 1} 


\vskip 1em
\noindent
which would set $\alpha=0.05~$\AA$^{-1}$, {\tt kmax1 = 6}, {\tt kmax2 =
6}. Once again one may check the accuracy by comparing the Coulombic
energy with the virial, as described above. The last two integers
specify, once again, the values of {\tt nhko} and {\tt nlatt}
respectively. (Note it is possible to set either of these to zero in
this case.)
 
Estimating the parameters required for a given simulation follows a
similar procedure as for the standard Ewald method (above), but is
complicated by the occurrence of higher orders of the convergence
functions. Firstly a suitable value for $\alpha$ may be obtained when
{\tt nlatt}=0 from the rule: $\alpha=\beta/r_{cut}$, where $r_{cut}$
is the largest real space cutoff compatible with a single MD cell and
$\beta$=(3.46,4.37,5.01,5.55) when {\tt nhko}=(0,1,2,3)
respectively. Thus in the usual case where {\tt nhko}=1, $\beta$=4.37.
When {\tt nlatt}$\ne$0, this $\beta$ value is multiplied by a factor
$1/(2*nlatt+1)$.

The estimation of {\tt kmax1,2} is the same as that for the standard
Ewald method above. Note that if any of these parameters prove to be
insufficiently accurate, \D{} will issue an error in the OUTPUT file,
and indicate whether it is the real or reciprocal space sums that is
questionable. 
